% ------------------------------------------------------------------------------
% The abstract should summarize the contents of the paper and should contain at
% least 70 and at most 150 words. It should be written using the \emph{abstract}
% environment. It must not contain references, as it may be used
% without the main article. It is acceptable, although not common, to identify
% work by author, abbreviation or RFC number (For example, "Our algorithm is
% based upon the work by Smith and Wesson."). Avoid use of "in this paper" in
% the abstract. What other paper would you be talking about here? Avoid general
% motivation in the abstract. You do not have to justify the importance of the
% Internet or explain what QoS is. Highlight not just the problem, but also the
% principal results. Many people read abstracts and then decide whether to
% bother with the rest of the paper. Since the abstract will be used by search
% engines, be sure that terms that identify your work are found there. In
% particular, the name of any protocol or system developed and the general area
% ("quality of service", "protocol verification", "service creation
% environment") should be contained in the abstract. Avoid equations and math.
% ------------------------------------------------------------------------------
% ------------------------------------------------------------------------------
% (1) Define the research area (which particular area are we focusing?).
% ------------------------------------------------------------------------------
\todoremark{area}\todo{one sentence about the research area (which particular area are we focusing?)}
% ------------------------------------------------------------------------------
% (2) Define the issue (what issue is getting to get solved?)
% ------------------------------------------------------------------------------
\todoremark{issue}\todo{one sentence about the issue (what issue is getting to get solved?)}
% ------------------------------------------------------------------------------
% (3) Shortcomings of existing solutions.
% ------------------------------------------------------------------------------
\todoremark{related work}\todo{one sentence about the shortcomings of existing solutions.}
% ------------------------------------------------------------------------------
% (4) Define the own approach.
% ------------------------------------------------------------------------------
\todoremark{approach}\todo{one sentence about the own approach}
% ------------------------------------------------------------------------------
% (5) How have we validated our results?
% ------------------------------------------------------------------------------
\todoremark{validation}\todo{one sentence about the validation of the results}
% ------------------------------------------------------------------------------
% (6) What have we shown / contributed? What is the result?
% ------------------------------------------------------------------------------
\todoremark{result}\todo{one sentence about the main contribution}
\todomid{about 150 characters}
\acresetall