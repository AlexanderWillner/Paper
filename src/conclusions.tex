\section{Conclusion and Future Work}\label{sec:conclusions}

%-------------------------------------------------------------------------------
%In general a short summarizing paragraph will do, and under no circumstances
%should the paragraph simply repeat material from the Abstract or Introduction.
%In some cases it's possible to now make the original claims more concrete,
%e.g., by referring to quantitative performance results.

%The further work material is important -- part of the value of a paper is
%showing how the work sets new research directions. (1) If you're actively
%engaged in follow-up work, say so. E.g.: "We are currently extending the
%algorithm to... blah blah, and preliminary results are encouraging." This
%statement serves to mark your territory. (2) Conversely, be aware that some
%researchers look to Future Work sections for research topics. 

%This section should summarize what has  been accomplished in the paper. Many
%readers will read only the Introduction and Conclusion of your paper. The
%Conclusion should be written so they can be understood by someone who has not
%read the main work of the paper.

%-------------------------------------------------------------------------------
%We have shown that\dots. The crucial results are\dots.  
%-------------------------------------------------------------------------------
We have shown that\dots. The crucial results are\dots.

%-------------------------------------------------------------------------------
%Recipient of the results/improvements. Who gaines provit from this?
%-------------------------------------------------------------------------------
Recipient of the results/improvements. Who gains profit from this?

%-------------------------------------------------------------------------------
%Further work: start with short-term, then long-term ojectives.
% -------------------------------------------------------------------------------
Further work: start with short-term, then long-term objectives.