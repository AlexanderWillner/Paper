\section{Introduction}\label{sec:introduction}
% ------------------------------------------------------------------------------
% The introduction should lead the reader to the work you've done. You should
% start with general statements and then get close to the core (zoom in). Every
% computer scientist should be able to understand what is paper is about. The
% introduction should follow the same structure as the abstract; but this time
% you should spend one paragraph for each item.
% ------------------------------------------------------------------------------
% ------------------------------------------------------------------------------
% (1) What is the problem? Motivation: broadly, what is problem area,
% why important? Open up the subject (the subject will be electromagnetic
% fields in cylindrical dielectric geometrics, adaptive arrays in packet radio,
% or whatever.)
% Introduce problem, outline the solution; the statement of the problem should
% include a clear statement why the problem is important (or interesting). In
% the case of a conference, make sure to cite the work of the PC co-chairs and
% as many other PC members as are remotely plausible, as well as from anything
% relevant from the previous two proceedings. In the case of a journal or
% magazine, cite anything relevant from last 2-3 years or so volumes. Avoid
% stock and cliche phrases such as "recent  advances in XYZ" or anything
% alluding to the growth of the Internet. Be sure that the introduction lets
% the reader know what this paper is about, not just how important your general
% area of research is. Readers won't stick with you for three pages to find out
% what you are talking about.
% The introduction must motivate your work by pinpointing the problem you are
% addressing and then give an overview of your approach and/ or contributions
% (and perhaps even a general description of your results). In this way, the
% intro sets up my expectations for the rest of your paper -- it provides the
% context, and a preview. Repeating the abstract in the introduction is a waste
% of space.
% ------------------------------------------------------------------------------
\sidenote{motivation}
\todotext{
One paragraph:
What is the problem? What is problem area. Open up the subject.
}

% ------------------------------------------------------------------------------
% (2) Why is it interesting and important? Why is it hard? (e.g., why do
% naive approaches fail?) Narrow down: what is problem you specifically
% consider? Describe the problem addressed in this paper.
% ------------------------------------------------------------------------------
\sidenote{importance}
\todotext{
One paragraph:
Why is it interesting and important? Why is it hard? (e.g., why do
naive approaches fail?) Narrow down: what is problem you specifically
consider? Describe the problem addressed in this paper.
}

% ------------------------------------------------------------------------------
% (3) Survey past work relevant to this paper. Why hasn't it been solved
% before (related work)? Or, what's wrong with previous proposed solutions? How
% does mine differ?
% ------------------------------------------------------------------------------
\sidenote{related work}
\todotext{
One paragraph:
Survey past work relevant to this paper. Why hasn't it been solved
before (related work)? Or, what's wrong with previous proposed solutions? How
does mine differ?
}

% ------------------------------------------------------------------------------
% (4) Describe the assumptions made in general terms, and state what results
% have been obtained. This gives the reader an initial overview of what problem
% is addressed in the paper and what has been achieved.
% ------------------------------------------------------------------------------
\sidenote{assumption}
In this paper, we assume ...
\todotext{
One paragraph: what are the assumptions
}

% ------------------------------------------------------------------------------
% (5) What are the key components of my approach and results? Also include any
% specific limitations. It's the most crucial paragraph, tell
% your elevator pitch: How is it different/better/relates to other work?
% Help the reviewer to get the scientific surplus value between all
% the motivation and basics.
% ------------------------------------------------------------------------------

% ------------------------------------------------------------------------------
\sidenote{contributions}
The main contribution of this paper is...
\todotext{
What are the key components of my approach and results? Also include any
specific limitations. It's the most crucial paragraph, tell
your elevator pitch: How is it different/better/relates to other work?
Help the reviewer to get the scientific surplus value between all
the motivation and basics.
}

% ------------------------------------------------------------------------------
% (6) How have we validated our results?
% ------------------------------------------------------------------------------
\sidenote{validation}
We have validated our approach by...
\todotext{
How have we validated our results?
}

% ------------------------------------------------------------------------------
% (7) ``The remainder of this paper is structured as follows...''. The last
% section must give an overview of the paper.
% ------------------------------------------------------------------------------
\sidenote{structure}
The remainder of the paper is structured as follows.
We give a brief overview of related work in the context of \todo{CHANGEME}
  in \sectionname~\ref{sec:relatedwork}.
In the subsequent \sectionname~\ref{sec:main} the
    \todo{CHANGEME} system is presented. 
The evaluation results obtained from \todo{CHANGEME} is discussed in
  \sectionname~\ref{sec:evaluation}.
Finally, we close giving some conclusions and considerations
  and describe future work in \sectionname~\ref{sec:conclusions}.
% ------------------------------------------------------------------------------
